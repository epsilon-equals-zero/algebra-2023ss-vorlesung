\section{Einführung}

\begin{definition}\label{def:unterkoerper}
    Sei $L$ ein Körper. Wir nennen $K \subseteq L$ einen \emph{Unterköper}\index{Unterkörper}, wenn $K$ mit den von $L$ auf $K$ eingeschränkten Operationen ein Körper ist\footnote{Da die Inversenbildung keine der Operationen des Körpers ist, ist es nicht ausreichend, wenn man für $K$ nur die Abgeschlossenheit unter allen Körperoperationen fordert. Diese Tatsache wird durch das Beispiel $\mathbb{Z}\subseteq \mathbb{Q}$ illustriert.}. Dafür schreiben wir auch $K \leq L$. In diesem Kontext heißt $L$ auch \emph{Oberkörper}\index{Oberkörper} von $K$.

    Da für jedes $x\in L^\times$ das multiplikative Inverse $x^{-1}$ eindeutig ist, ist der Schnitt von Unterkörpern wieder ein Unterkörper. Damit ist durch
    $$ \bigcap \{ U \mid U\le L \} $$
    ein Unterkörper von $L$ gegeben, welchen wir den \emph{Primkörper}\index{Primkörper} von $L$ nennen.

    Sei $K \leq L, S \subseteq L$ so definieren wir die \emph{Körpererweiterung von $K$ um $S$}\index{Körpererweiterung} durch
    $$ K(S) := \bigcap \{ U \mid K \leq U \leq L, U \supseteq S \}. $$
    Ist $S = \{ \alpha_1, \hdots, \alpha_n \}$, so schreiben wir auch $K(\alpha_1, \hdots, \alpha_n)$.
    
    Analog ist die \emph{Ringerweiterung von $K$ um $S$}\index{Ringerweiterung} definiert:
    $$ K[S] := \bigcap \{ U \mid U \;\mathrm{Ring}\land K \subseteq U \subseteq L, U \supseteq S \}. $$
    Ist $S = \{ \alpha_1, \hdots, \alpha_n \}$, so schreiben wir auch $K[\alpha_1, \hdots, \alpha_n]$.
\end{definition}

\begin{remark}
    Man sieht sofort, dass beispielsweise $\mathbb{R}(i) = \mathbb{C}$ gilt.
\end{remark}

\begin{remark}
    Für einen Körper $K$ ist $K$ ein Unterkörper von $K(x)$.
\end{remark}

\begin{definition}
    Ein Körper $K$ heißt \emph{Primkörper}\index{Primkörper}\index{Köper!Prim-}, wenn $K$ keine echten Unterkörper hat.
\end{definition}

\begin{remark}
    Sei $L$ ein Körper und $K$ der Primkörper von $L$. Dann ist $K$ ein Primkörper.
\end{remark}

\begin{theorem} \label{theorem:primkoerper}
    Sei $K$ ein Primkörper.
    \begin{itemize}
        \item Ist $\chara K = 0$, so ist $K \cong \mathbb{Q}$.
        \item Ist $\chara K = p \in \mathbb{P}$, so ist $K \cong \mathbb{Z}_p$.
    \end{itemize}
\end{theorem}

\begin{proof}
    Wir beweisen zunächst die erste Behauptung. Sei $K$ ein Körper mit Charakteristik $0$.
    Dann definieren wir eine Abbildung $\varphi:\mathbb{Q}\to K$ durch $\varphi(\pm\frac{a}{b})=\pm\frac{\overbrace{1+\ldots+1}^{a}}{\underbrace{1+\ldots+1}_{b}}$,
    mit $a\in \mathbb{N}$, $b\in\mathbb{N}\setminus\{0\}$. Diese Abbildung ist wohldefiniert,
    da der Nenner laut Voraussetzung niemals $0$ wird und sie unabhängig von der Wahl der Repräsentanten ist
    (kürzbare Ausdrücke in $\mathbb{Q}$ sind auch in $K$ kürzbar).
    Wie man leicht sieht, ist die Abbildung ein Homomorphismus. Und $\varphi$ ist außerdem injektiv, denn gilt
    $\varphi(\frac{a}{b})=0$, so folgt sofort $\frac{a}{b}=0$, da wir $\mathrm{char}(K)=0$ vorausgesetzt haben.
    Da $\varphi(\mathbb{Q})$ einen Unterkörper von $K$ darstellt und $K$ aber ein Primkörper ist, folgt die Surjektivität von $\varphi$,
    also $K\cong \mathbb{Q}$.

    Für die zweite Behauptung definieren wir zunächst $\varphi:\mathbb{Z}\to K$ durch $\varphi(\pm k)=\pm(\overbrace{1+\ldots+1}^{k})$ für $k\in \mathbb{N}$. Klarerweise ist $\ker\varphi=p\mathbb{Z}$ und nach dem Homomorphiesatz gilt $\mathbb{Z}_p=\mathbb{Z}/\ker\varphi\cong \phi(\mathbb{Z})$. Daher ist $\varphi(\mathbb{Z})$ ein Körper und es folgt $K=\varphi(\mathbb{Z})$.
\end{proof}